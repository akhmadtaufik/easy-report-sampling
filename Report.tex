\documentclass{article}

\usepackage[top=1in, bottom=1in, left=1in, right=1in]{geometry}
\usepackage{amsmath} %untuk menggunakan notasi matematika
\usepackage{parskip}
\usepackage{biblatex} %untuk menambahkan BibliTex
\usepackage{indentfirst}
\usepackage{caption}

\addbibresource{ref.bib} % menambahkan file .bib sebagai sumber referensi

\setcounter{secnumdepth}{0} %menghilangkan penomoran pada section dan subsection

\begin{document}

\title{Analisa Pasar Mengenai Peminatan Masyarakat Bandung Raya Terhadap Re-aktivasi Jalur Kereta Komuter Bandung - Tanjung Sari}
\author{Akhmad Taufik Ismail}
\maketitle

\section{Latar Belakang dan Objektif} \par

Kota-kota besar di negara-negara Asia-Pasifik telah mengalami perubahan yang signifikan dalam dua hingga tiga dekade terakhir karena urbanisasi. Penyebab urbanisasi berbeda-beda di setiap negara, tetapi pada dasarnya disebabkan oleh ketidakseimbangan spasial, seperti perbedaan jumlah penduduk dan ekonomi. Urbanisasi seringkali menyebabkan masalah sosial, ekonomi, dan perumahan di kota dan pedesaan \cite{Rustiadi2009}. Negara berkembang telah mengalami urbanisasi sejak tahun 1950 sebagai akibat dari perubahan ekonomi yang dihasilkan dari reorganisasi produksi, tenaga kerja, keuangan, penyediaan layanan, dan persaingan global, yang ditandai dengan pertumbuhan penduduk di kota-kota besar. Di Indonesia, selain Jakarta, kota-kota besar lainnya juga memiliki peran penting dalam pembangunan wilayah sebagai pusat kegiatan ekonomi yang memiliki hubungan dengan kota-kota besar lainnya, salah satunya adalah di Daerah Metropolitan Bandung (BMA) atau yang juga dikenal sebagai Kawasan Cekungan Bandung atau Bandung Raya \cite{Fuadina2021}. \par

Bandung Raya, yang juga dikenal sebagai Cekungan Bandung, adalah salah satu wilayah metropolitan terbesar di Indonesia yang terletak di antara Jakarta dan Surabaya. Pada tahun 2021, wilayah ini memiliki populasi lebih dari 9 juta jiwa \cite{BPS_Kota_Bandung_2022} \cite{BPS_Kota_Cimahi_2022} \cite{BPS_Kabupaten_Bandung_2022} \cite{BPS_Kabupaten_Bandung_Barat_2022} \cite{BPS_Kabupaten_Sumedang_2022} dan menempati peringkat ketiga dalam hal jumlah penduduk di negara ini. Kedua wilayah metropolitan ini, yaitu Jakarta dan Bandung, saat ini mengalami ekspansi perkotaan dan terus berkembang menjadi semakin terkait melalui koridor perkotaan yang menghubungkan Serang dan Cikampek serta Bandung dan Jakarta. Dalam hal ini, wilayah ini dikenal dengan nama Jakarta-Bandung Mega-Urban Region atau JBMUR. Wilayah ini meliputi Kota Bandung, Kabupaten Bandung, Kabupaten Bandung Barat, Kota Cimahi, dan 4 kecamatan di Kabupaten Sumedang yang ditetapkan sebagai Kawasan Strategis Nasional dalam aspek ekonomi oleh Peraturan Presiden Nomor 26 Tahun 2008. \par

Tujuan dari Re-aktivasi jalur kereta Bandung-Tanjung Sari adalah untuk meningkatkan konektivitas transportasi publik di wilayah Bandung Raya. Kondisi transportasi publik saat ini relatif buruk, dengan hanya tersedia dua rute bus rapid dan angkot yang menghubungkan kedua daerah tersebut. Hal ini menyebabkan kesulitan bagi masyarakat yang bekerja atau belajar di kawasan Tanjung Sari dalam mencapai tujuan mereka. Terdapat beberapa universitas besar di Kecamatan Jatinangor, Kabupaten Sumedang yang membuat potensi pasar untuk jalur kereta komuter ini cukup tinggi. Oleh karena itu, dibutuhkan analisis pasar yang tepat untuk mengetahui tingkat minat masyarakat dalam menggunakan jalur kereta komuter baru ini. \par

Analisis pasar ini akan memberikan informasi mengenai segmentasi pasar yang tepat, kebutuhan masyarakat dan tingkat persaingan dalam industri transportasi publik di wilayah Bandung Raya. Dengan demikian, pengambilan keputusan yang dilakukan dapat lebih tepat dan sesuai dengan kebutuhan masyarakat, sehingga dapat memberikan dampak positif bagi pengembangan transportasi publik di wilayah Bandung Raya. Analisis pasar ini juga dapat digunakan untuk mengetahui tingkat minat masyarakat dalam menggunakan jalur kereta komuter baru yang akan dire-aktivasi dari Stasiun Rancaekek ke Tanjung Sari. Untuk mengestimasi demand pengguna jalur baru tersebut, akan dilakukan metode multi stage sampling.

\section{Rangacangan Sampling} \par

\subsection{Populasi}
Wilayah Bandung Barat terdiri dari Kota Bandung, Kota Cimahi, Kabupaten Bandung, Kabupaten Bandung Barat dan Kabupaten Sumedang. Populasinya terbagi menjadi tiga tahap, yaitu:

\begin{itemize}
    \item Tahap 1 : Seluruh kecamatan di wilayah Bandung Raya;
    \item Tahap 2 : Seluruh desa atau kelurahan di wilayah Bandung Raya;
    \item Tahap 3 : Penduduk usia produktif di kelurahan.
\end{itemize}
    
\subsection{Populasi Target}
Seluruh penduduk usia produktif (15-34 tahun) di kecamatan yang dilewati jalur kereta Padalarang - Cicalengka dan jalur cabang Rancaekek - Tanjung Sari yang akan di re-aktivasi.
\subsection{Kerangka Sampel}
\begin{itemize}
    \item Total populasi penduduk Bandung Raya adalah 9,863,738 jiwa.
    \item Luas Wilayah Bandung Raya adalah 4,930.03 kilometer persegi.
    \item Total kecamatan yang berada di wilayah Bandung Raya sebanyak 107 kecamatan.
    \item Total desa atau kelurahan yang berada di Wilayah Bandung Raya sebanyak 905 desa atau kelurahan.
    \item Rasio usia produktif (15-34 tahun) pada populasi sebesar 30\%.
    \item Total kecamatan pada populasi target adalah 28 kecamatan.
    \item Total desa atau kelurahan pada populasi target adalah 201 desa atau kelurahan.
    \item Total penduduk pada populasi target adalah 930.650 jiwa.
    \item Proporsi pengguna transportasi publik di Kota Bandung berkisar 18\%.
\end{itemize}

\subsection{Unit Sampling}
Untuk penelitian ini, sampel akan diambil berdasarkan unit-unit berikut:
\begin{itemize}
\item \textbf{Unit Pertama:} Kecamatan yang dilewati jalur kereta lokal Padalarang-Cicalengka.
\item \textbf{Unit Kedua:} Kelurahan atau desa di masing-masing kecamatan.
\item \textbf{Unit Ketiga:} Penduduk dalam masing-masing stratum.
\end{itemize}

\subsection{Unit Observasi}
Pada penelitian ini, unit observasi adalah penduduk, karena mereka yang akan ditanya dan memberikan pandangannya mengenai rencana re-aktivasi jalur kereta lokal Rancaekek-Tanjungsari.

\subsection{Unit Analisis}
Unit analisis dalam penelitian ini adalah penduduk usia produktif (15-34 tahun) yang berada dalam desa atau kelurahan yang dipilih sebagai unit observasi.

\subsection{Karakteristik Yang Diteliti}
Dalam analisa kali ini akan mengukur tingkat peminatan masyarakat untuk menggunakan kereta komuter Bandung - Tanjung Sari yang akan dire-aktivasi.

\subsection{Nilai Karakteristik Yang Diteliti}
Oleh karena tingkat peminatan yang akan diteliti maka nilai karakteristik yang akan dicari adalah estimasi demmand penumpang dan estimasi rata-rata penumpang perhari.

\subsection{Metode Sampling}

Multi stage sampling adalah metode sampling yang cocok untuk digunakan dalam kasus ini karena populasi yang akan diambil sampelnya cukup besar dan tersebar di wilayah yang luas. Metode ini memungkinkan untuk mengambil sampel secara efisien dengan mengambil sampel pada beberapa tahap atau tingkatan. Dalam kasus ini, tahap pertama adalah pemilihan kecamatan, tahap kedua adalah pemilihan desa atau kelurahan, dan tahap ketiga adalah pemilihan penduduk usia produktif. \par

Multi stage sampling juga cocok digunakan karena dapat mengurangi bias yang mungkin terjadi dalam pemilihan sampel dan meningkatkan representativitas sampel yang diambil.

Selain itu, multi stage sampling juga cocok digunakan karena dapat mengurangi biaya yang dibutuhkan untuk mengambil sampel, khususnya dalam hal ini yang diteliti adalah tingkat peminatan masyarakat yang cukup spesifik dan terbatas pada wilayah yang diteliti yaitu kecamatan, desa atau kelurahan, dan penduduk usia produktif.

\section{Langkah-langkah Pengambilan Sampel}

Proses pengambilan sampel menggunakan metode multi stage sampling dalam kasus ini dapat dilakukan dengan beberapa tahap sebagai berikut:

\begin{enumerate}
    \item Tahap 1: Pemilihan kecamatan. 
    
    \par\noindent
    Pada tahap ini, akan memulai pengambilan sampel. Pertama-tama harus menentukan target populasi dan ukuran populasi. Target populasi dalam hal ini adalah Kecamatan yang dilewati oleh jalur kereta lokal Bandung Raya, baik jalur utama Padalarang - Cicalengka maupun jalur yang akan dire-aktivasi, yaitu: Rancaekek - Tanjung sari yang merupakan jalur cabang. Setelah melakukan Pra-survey diketahui terdapat 28 kecamatan yang dilewati oleh jalur utama maupun cabang, 28 kecamatan ini merupakan Primary Sampling Unit Populasi atau biasa dinotasikan dengan $N$.

    
    \par\noindent
    Daftar kecamatan target populasi dapat dilihat pada tabel berikut:
    \begin{table}[ht]
        \centering
        \caption{Daftar Kecamatan Target Populasi}
        \label{tabel:psu_populasi}
        \begin{tabular}{|p{3cm}|p{3cm}|p{3cm}|p{3cm}|}
        \hline
        Andir & Antapani & Bandung Wetan & Batununggal \\ \hline
        Bojongloa Kaler & Buah Batu & Cicendo & Cinambo \\ \hline
        Gedebage & Kiaracondong & Lengkong & Panyileukan \\ \hline
        Rancasari & Regol & Sumur Bandung & Cimahi Selatan \\ \hline
        Cimahi Tengah & Cimahi Utara & Cicalengka & Cileunyi \\ \hline
        Rancaekek & Cikalong Wetan & Cipatat & Cisarua \\ \hline
        Ngamprah & Padalarang & Jatinangor & Tanjungsari \\ \hline
        \end{tabular}
\end{table}

Setelah mengetahui PSU populasi, langkah selanjutnya adalah mencari Primary Sampling Unit sampel (PSU) dengan notasi $n$. Nilai $n$ dapat ditemukan dengan menggunakan rumus berikut:
\begin{align}
    n = \frac{\frac{z^{2}p(1-p)}{e^{2}}}{1 + \left(\frac{z^{2}p(1-p)}{e^{2}N}\right)}
\end{align}
di mana:
\begin{itemize}
    \item $p$ = Proporsi
    \item $e$ = Margins of Error
    \item $z$ = Z-statistik
    \item $N$ = Jumlah Populasi
\end{itemize}

\par\noindent
Berdasarkan persamaan (1) maka didapatkan PSU sampel sebanyak 8 kecamatan, dengan Confidence Interval 75\% dan Margin of Errors sebesar 25\%. Daftar Kecamatan yang di ambil dapat dilihat pada tabel \ref{tabel:psu_sampel}.
\begin{table}[ht]
    \centering
    \caption{Daftar Kecamatan yang Diambil}
    \label{tabel:psu_sampel}
    \begin{tabular}{|p{3cm}|p{3cm}|}
    \hline
    Tanjungsari & Cimahi Tengah  \\ \hline
    Cimahi Selatan & Jatinangor \\ \hline
    Bojongloa Kaler & Rancaekek  \\ \hline
    Sumur Bandung & Cileunyi  \\ \hline
    \end{tabular}
\end{table}
    
    \item Tahap 2: Pemilihan desa atau kelurahan. 
    
    \par\noindent
    Setelah mengetahui jumlah kecamatan yang diambil, langkah berikutnya adalah memilih kelurahan secara acak dari ke-8 kecamatan tersebut. Tahap ini masih menggunakan teknik cluster sampling, sehingga jumlah desa/kelurahan pada tiap kecamatan yang dipilih menjadi Primary Sampling Unit Populasi. Jumlah desa/kelurahan setiap kecamatan berbeda, sehingga jumlah $N$ juga akan berbeda. Persamaan (1) masih digunakan untuk menentukan jumlah sampel, sehingga $n$ akan dihitung untuk setiap kecamatan.

    Tabel \ref{tabel:psu_kel_1} adalah jumlah $n$ dan daftar kelurahan pada Kecamatan Cileunyi, Sumur Bandung, Cimahi Tengah, Cimahi Selatan dan Bojongloa Kaler. Sedangkan Tabel \ref{tabel:psu_kel_2} merupakan jumlah $n$ kelurahan di Kecamatan Jatinangor, Rancaekek, dan Tanjungsari.
    \begin{table}[ht]
        \centering
        \caption{Jumlah Sampel ($n$) Kelurahan di Kecamatan Cileunyi, Sumur Bandung, Cimahi Tengah, Cimahi Selatan dan Bojongloa Kaler.}
        \label{tabel:psu_kel_1}
        \begin{tabular}{|p{3cm}|p{2.4cm}|p{3cm}|p{3cm}|p{3cm}|}
        \hline
        \textbf{Bojongloa Kaler} & \textbf{Cileunyi} & \textbf{Cimahi Selatan} & \textbf{Cimahi Tengah} & \textbf{Sumur Bandung} \\ \hline
        Kopo & Cileunyi Kulon & Cibeber & Cimahi & Merdeka \\ \hline
        Babakan Asih & Cileunyi Wetan & Leuwi Gajah & Setiamanah & Braga \\ \hline
        Suka Asih & Cinunuk  & Cibeureum & Baros & Kebon Pisang \\ \hline
        \end{tabular}
    \end{table}

    \begin{table}[ht]
        \centering
        \caption{Jumlah Sampel ($n$) Kelurahan di Kecamatan Jatinagor, Rancaekek, Tanjungsari}
        \label{tabel:psu_kel_2}
        \begin{tabular}{|p{3cm}|p{3cm}|p{3cm}|}
        \hline
        \textbf{Jatinangor} & \textbf{Rancaekek} & \textbf{Tanjungsari} \\ \hline
        Cisempur & Rancaekek Kulon & Jatisari \\ \hline
        Cileles & Tegal Sumedang & Margaluyu \\ \hline
        Jatimukti & Sukamulya  & Raharja  \\ \hline
        Mekargalih & Nanjung Mekar  & Cijambu  \\ \hline
        \end{tabular}
    \end{table}

    \item Tahap 3: Pemilihan penduduk usia produktif berdasarkan jenis pekerjaan. 
    \par\noindent
    Pada tahap ini yang pertama dilakukan adalah  menentukan jenis pekerjaan dan usia produktif sebagai kriterium stratifikasi. Kriterium usia produktif adalah penduduk yang berusia 15-34 tahun dan jenis pekerjaan meliputi Dosen/Guru, Karyawan BUMN/BUMD, Karyawan Swasta, PNS, Pedagang/Wiraswasta, Pekerja Medis, dan Pelajar/Mahasiswa.

    Selanjutnya adalah menentukan jumlah sampel berdasarkan populasi penduduk pada tahap kedua. Setelah melakukan perhitungan pada kelima wilayah Bandung Raya didapatkan proporsi penduduk usia produktif dibanding dengan total populasi berada pada rentang $30-35\%$, maka akan diambil asumsi $30\%$ dari populasi adalah penduduk berusia produktif. Untuk menentukan jumlah sampel akan digunakan Slovin Formula Nilai $n$ dapat ditemukan dengan menggunakan rumus berikut:
    \begin{align}
        n = \cfrac{N}{1 + Ne^{2}}
    \end{align}

    dimana:
    \begin{itemize}
        \item $n$ = Jumlah sampel yang diinginkan
        \item $e$ = Margin of Error
        \item $N$ = Jumlah Populasi
    \end{itemize}

    Langkah selanjutnya adalah menentukan ukuran sampel dari setiap stratum dengan menggunakan proportional allocation stratified sampling. Metode ini digunakan karena  memastikan bahwa setiap stratum memiliki proporsi sampel yang sama dengan proporsi populasi pada stratum tersebut, sehingga menjamin representatifitas sampel dan tidak membahayakan validitas hasil penelitian. Rumus yang digunakan untuk mencari jumlah sampel untuk tiap stratum ($n_{h}$) sebagai berikut:
    \begin{align}
        n_{h} = n \cfrac{N_{h}}{N}
    \end{align} 

    dimana:
    \begin{itemize}
        \item $n_{h}$ = Jumlah sampel pada strata $h$
        \item $n$ = Jumlah sampel
        \item $N_{h}$ = Jumlah populasi pada strata $h$
        \item $N$ = Jumlah Populasi 
    \end{itemize}

    Berdasarkan persamaan (2) dan (3) maka hasil jumlah sampel ($n$) yang diperoleh adalah 400 dengan margin of error sebesar $5\%$. Untuk jumlah sampel pada tiap stratum dapat dilihat pada tabel \ref{tabel:sample_stratum} .
    \begin{table}[ht]
        \centering
        \caption{Jumlah Populasi ($N_{h}$) dan Sampel ($n_{h}$) pada setiap stratum}
        \label{tabel:sample_stratum}
        \begin{tabular}{|p{6cm}|p{3cm}|p{3cm}|}
        \hline
        \textbf{Strata} & \textbf{$N_{h}$} & \textbf{$n_{h}$} \\ \hline
        Dosen/Guru & 1,540 & 5 \\ \hline
        Karyawan BUMN/BUMD & 718 & 2 \\ \hline
        Karyawan Swasta & 23,203  & 90  \\ \hline
        PNS & 2,669  & 10  \\ \hline
        Pedagang/Wiraswasta & 26,899	 & 104  \\ \hline
        Pekerja Medis & 410 & 1  \\ \hline
        Pelajar/Mahasiswa & 47,228  & 187  \\ \hline
        \end{tabular}
    \end{table}

\end{enumerate}

\subsection{Metode Pengumpulan Data}

Metode pengambilan data dalam kasus ini akan menggunakan beberapa metode, seperti:
\begin{itemize}
    \item Survei: melakukan wawancara atau menyebarkan kuesioner kepada responden untuk mengetahui tingkat peminatan masyarakat terhadap moda transportasi yang akan dire-aktivasi.
    \item Observasi: melakukan pengamatan secara langsung.
\end{itemize}

Pada kasus ini, metode survei dan observasi dapat digunakan untuk mengumpulkan data yang dibutuhkan. Survei dapat digunakan untuk mengetahui tingkat peminatan masyarakat terhadap moda transportasi publik, sedangkan observasi dapat digunakan untuk mengetahui estimasi demand penumpang dan rata-rata penumpang harian.

Untuk survei, beberapa pertanyaan yang mungkin ditanyakan adalah:
\begin{itemize}
    \item Jenis transportasi yang digunakan saat ini dalam perjalanan ke tempat kerja atau kampus.
    \item Preference dan alasan pemilihan transportasi yang digunakan saat ini.
    \item Kemudahan dan kesulitan dalam menggunakan transportasi publik.
    \item Frekuensi penggunaan transportasi publik
    \item Perbandingan harga transportasi publik dengan transportasi pribadi.
    \item Tingkat peminatan untuk pindah ke moda transportasi publik di masa depan.
\end{itemize}

Sedangkan untuk observasi, ada beberapa hal yang harus diobservasi secara langsung, yaitu:
\begin{itemize}
    \item Kondisi jalur kereta saat ini: Melihat secara langsung terhadap kondisi jalur kereta yang ada saat ini, seperti kondisi rel, jembatan, stasiun, dan peralatan lainnya. Hal ini akan membantu dalam menentukan kebutuhan perbaikan.
    \item Kondisi lingkungan sekitar jalur: Melihat kondisi lingkungan sekitar jalur kereta, seperti kondisi jalan, jembatan, dan fasilitas-fasilitas lainnya yang diperlukan untuk aksesibilitas.
    \item Kondisi transportasi publik saat ini: Melakukan observasi terhadap kondisi transportasi publik saat ini, seperti jumlah kendaraan, frekuensi keberangkatan, jadwal, dan tingkat kepadatan.
    \item Kondisi aksesibilitas transportasi publik: Melakukan observasi terhadap kondisi aksesibilitas transportasi publik, seperti jumlah halte, jumlah jalur, dan tingkat aksesibilitas ke jalur yang akan dire-aktivasi.
\end{itemize}

Semua poin observasi tersebut akan membantu dalam perencanaan re-aktivasi jalur kereta lokal yang sesuai dengan kebutuhan dan kondisi masyarakat sekitar.

\subsection{Metode Estimasi Parameter}

Pertama-tama adalah melakukan estimasi rata-rata pada tiap stratum, hal ini untuk melihat berapa banyak keinginan untuk pindah ke transportasi publik yang direaktivasi. Dalam metode stratified sampling, populasi dibagi menjadi beberapa stratum yang mewakili subkelompok dalam populasi. Setiap stratum dianalisis secara terpisah dan hasilnya kemudian digabungkan untuk mendapatkan estimasi rata-rata secara keseluruhan. Rumus yang digunakan sebagai berikut \cite{thompson_2012}:
\begin{align}
    \bar{y}_{h} = \cfrac{1}{n_{h}} \sum_{i=1}^{n_{h}} y_{hi}
\end{align} 

di mana:
\begin{itemize}
    \item $\bar{y}_{h}$ = Estimasi rata-rata pada strata $h$
    \item $n_{h}$ = Jumlah sampel pada strata $h$
    \item $y_{hi}$ = Nilai karakteristik $y$ pada elemen ke-$i$ strata $h$
\end{itemize}

Rata-rata dari masing-masing stratum ini kemudian dapat digunakan untuk mengestimasi rata-rata populasi secara keseluruhan. Estimasi rata-rata pada strata memberikan estimasi yang lebih akurat daripada estimasi rata-rata secara acak karena memastikan bahwa masing-masing stratum terwakili dalam sampel yang diambil. Rumus untuk menghitung estimasi rata-rata pada populasi sebagai berikut:
\begin{align}
    \bar{y}_{st} = \cfrac{1}{N} \sum_{h=1}^{L} N_{h} \bar{y}_{h}
\end{align}

di mana:
\begin{itemize}
    \item $\bar{y}_{st}$ = rata-rata tertimbang rata-rata sampel dari karakteristik yang diminati dari setiap strata.
    \item $\bar{y}_{h}$ = Estimasi rata-rata pada strata $h$
    \item $N$ = Jumlah populasi.
    \item $N_{h}$ = Jumlah populasi pada strata $h$
\end{itemize}

Selanjutnya adalah mengitung total demand penumpang menggunakan rumus estimasi total populasi sebagai berikut:
\begin{align}
    \hat{\tau}_{st} = \sum_{h=1}^{L} N_{h} \bar{y}_{h}
\end{align}

Estimasi varians dari estimator total populasi adalah ukuran varians dari estimator yang digunakan untuk mengestimasi total populasi. Variance ini mengukur seberapa besar estimasi yang diberikan berbeda dari nilai sebenarnya. Semakin kecil varians dari estimator, semakin akurat estimasi yang diberikan.

Estimator populasi total adalah jumlah semua elemen dalam populasi, dan estimasi dari estimator ini dapat ditemukan dengan menjumlahkan semua sampel yang diambil dari populasi. Varians dari estimator total populasi dapat ditemukan dengan menggunakan rumus berikut:
\begin{align}
    \widehat{\text{var}}(\hat{\tau}_{st}) = \sum_{h=1}^{L} 
    N_{h} (N_{h}-n_{h})
    \cfrac{s_{h}^{2}}{n_{h}}
\end{align}

dengan:
\begin{align}
    s_{h}^{2} = \cfrac{1}{n_{h}-1}
    \sum_{i=1}^{n_{h}}
    \left ( y_{hi} - \bar{y}_{h} \right )^{2}
\end{align}

Confidence interval adalah rentang nilai yang memuat nilai sebenarnya dari parameter populasi dengan tingkat keyakinan tertentu. Tingkat keyakinan ini biasanya dinyatakan sebagai persentase, misalnya 95%.

Berdasarkan estimasi varians dari estimator total populasi, confidence interval dapat ditemukan dengan menggunakan rumus berikut:
\begin{align}
    \hat{\tau}_{st} \pm d
\end{align}

dengan nilai $d$ ditemukan dengan rumus:
\begin{align}
    \text{d} = z_{\alpha/2} \sqrt{\widehat{\text{var}}(\hat{\tau}_{st})}
\end{align}

di mana:
\begin{itemize}
    \item $z_{\alpha/2}$ = Nilai z-statistik dalam tingkat keyakinan tertentu.
\end{itemize}

Confidence interval memberikan ukuran besar marga kesalahan dari estimasi. Semakin kecil varians dari estimator total populasi, semakin sempit confidence interval. Confidence interval juga memberikan ukuran besar tingkat keyakinan dari estimasi. Semakin besar tingkat keyakinan, semakin lebar confidence interval. Tingkat keyakinan dapat ditingkatkan dengan menambah ukuran sampel atau dengan menggunakan estimator yang lebih akurat.


\section{Hasil dan Diskusi}
Pada desain sampling ini, metode multi-stage sampling tiga tahap digunakan. Tahap pertama menggunakan Cluster Sampling, yang terdiri dari $N=28$ kecamatan. Setelah menggunakan persamaan (1), didapat $n$ dengan confidence interval sebesar $90\%$ dan margin of error sebesar $25\%$ sebanyak $8$ kecamatan. Detail kecamatan yang diambil dapat dilihat pada Tabel 2.

Selanjutnya, pada tahap kedua, metode Cluster Sampling digunakan lagi. Maka $M_i=8$ kecamatan diambil, namun nilai $N_i$ (kelurahan) pada setiap $M_i$ bervariasi maka nilai $n_i$ yang dihitung pun akan bervariasi. Untuk menentukan nilai $n_i$, persamaan (1) digunakan dan didapat $n$ dengan confidence interval sebesar $75\%$ dan margin of error sebesar $25\%$, seperti yang terlihat pada Tabel 3 dan Tabel 4.

Selanjutnya, pada tahap ketiga, metode Stratified Sampling digunakan. Populasi kelurahan yang terpilih di tahap kedua, yang ditandai dengan $N$, berjumlah 102,670 jiwa. Kemudian, menggunakan persamaan (2), jumlah sampel ($n$) ditentukan sebanyak 400 orang. Populasi $N$ akan dibagi menjadi beberapa stratum, yaitu: Dosen/Guru, Karyawan BUMN/BUMD, Karyawan Swasta, PNS, Pedagang/Wiraswasta, Pekerja Medis, dan Pelajar/Mahasiswa. Jumlah penduduk berdasarkan stratum dapat ditandai dengan $N_h$. Menggunakan persamaan (3), jumlah sampel ($n_h$) pada setiap stratum ditentukan, dan hasil populasi dan sampel pada setiap stratum dapat dilihat pada Tabel 5.

Rancangan sampling yang digunakan menggunakan metode multi-stage sampling dengan tiga tahap. Tahap pertama menggunakan cluster sampling, tahap kedua menggunakan cluster sampling lagi, dan tahap ketiga menggunakan stratified sampling. Keunggulan rancangan sampling tersebut adalah:
\begin{enumerate}
    \item Menggunakan metode multi-stage sampling memungkinkan pengambilan sampel dengan lebih akurat dan efisien.
    \item Menggunakan cluster sampling pada tahap pertama dan kedua memungkinkan pengambilan sampel dengan lebih mudah dan efisien, karena tidak memerlukan data populasi seluruh wilayah Bandung Raya.
    \item Menggunakan stratified sampling pada tahap ketiga memungkinkan pengambilan sampel yang lebih representatif, karena memastikan bahwa setiap stratum dalam populasi terwakili dengan proporsional.
    \item Menentukan jumlah sampel dengan menggunakan persamaan yang sesuai memastikan bahwa jumlah sampel yang diambil memenuhi spesifikasi yang ditentukan seperti confidence interval dan margin of error.
\end{enumerate}

Rancangan sampling tersebut memiliki beberapa kekurangan yang perlu diperhatikan, di antaranya:
\begin{enumerate}
    \item Data yang diambil mungkin tidak representatif karena ditentukan dengan metode Cluster Sampling. Dalam hal ini, hasil sampling dapat dipengaruhi oleh pemilihan kelurahan atau desa yang tidak merata.
    \item Metode Stratified Sampling membatasi jumlah sampel yang diambil dan hanya mengambil sampel dari stratum tertentu. Dalam hal ini, hasil sampling mungkin tidak representatif karena hanya mengambil sampel dari beberapa stratum saja.
    \item Keterbatasan dalam memperhitungkan metodologi dan alat pengambilan data yang digunakan sehingga mungkin ada kemungkinan adanya bias.
\end{enumerate}

Untuk memperbaiki kekurangan tersebut, dibutuhkan evaluasi dan perbaikan metodologi dan alat pengambilan data yang lebih tepat dan akurat.

\printbibliography %menampilkan daftar referensi

\end{document}
